% Example file for 130th Convention paper
\documentclass{aes130}
\hyphenation{Post-Script}

 
\authors{John Author\aff{1}, 
         Jane Quincy-Author\aff{1}, 
         and James Researcher\aff{2}}

\affiliation[1]{BigCity College, City, ST, 00000, Country}
\affiliation[2]{Smallville State Technical Institute, Smallville, XR,
99999, Country}

\correspondence{Jane Quincy-Author}{jane\_author@snailmail.qzl}

\lastnames{Author, Quincy-Author, Researcher}

\title{The Title of this Paper is not\\ ``The Title of this Paper''}

\shorttitle{Abbreviated Paper Title}


\begin{abstract}
An informative and self-contained abstract of about 120 words must be
provided. An informative and self-contained abstract of about 120 words
must be provided. An informative and self-contained abstract of about
120 words must be provided. An informative and self-contained abstract
of about 120 words must be provided. An informative and self-contained
abstract of about 120 words must be provided.
\end{abstract}

\begin{document}

\maketitle % MANDATORY! 


\section{MAJOR HEADING}
This \LaTeX\ file, which should be compatible with any PC, Unix, or
Macintosh-based \LaTeX\ system, is intended to present a consistent style so
that all authors use the same front-page header, two-column format,
and headers and footers. It should be possible simply to replace this text
with your prepared text.


\textbf{Note: You may use any word-processing or page-layout program.}
The AES paper-submission site will convert your file to a PDF file;
you can also submit a PDF. If you do submit a PDF, make sure to embed
your fonts and graphics but do not use password protection.
\textbf{Your PDF should have only Type 1 fonts. Please try to keep the
size of the final PDF file below 1 MB}. If it is larger try using
lower-resolution graphics to reduce the size of the file.

Papers for AES conventions are not edited by the AES editorial
staff. Papers will be published in print, on CD-ROM, and on the AES
web site exactly as submitted. It is therefore important that authors
prepare their manuscripts in accordance with these guidelines.

Major headings (see above) are in capital letters (upper case),
Helvetica font, 10 point size, bold style, left justified. All headings
are preceded by a blank line. Body text, like this, is Times Roman
font, 10 point size, plain style. It should be fully justified to be
flush left and flush right. 

\subsection{Minor Heading}
Minor headings use both upper-case and lower-case letters and, like
the major headings, Helvetica font, 10 point size, bold style, left
justified.

\subsubsection{Example}
This is an example subsubsection set using the \verb|\subsubsection|
command. 


\begin{table*}
\begin{center}
\begin{tabular}{|c|l|r|}
\hline
234093241&23402312&3432829807434\\
2398234&423403290&123144298\\
2340243012597398&1245987533&24982499\\
\hline
\end{tabular}
\caption{This is a two-column table.}
\end{center}
\end{table*}

\section{CONTENT}
To ensure that convention papers are consistent with the goals of the
AES, the following guidelines shall be considered by authors.

Technical content shall be accurate. It should cite original work or
review previous work giving proper credit.

If the paper describes a product, the content should dwell on the
technical aspects of the equipment (circuitry, layout, specifications,
functions, and applications). 

Company logos shall not be used. Manufacturers name and model names or
numbers shall not be used in titles and abstracts, and should be kept
to a minimum in the text (generic descriptions should be used).

The trademark symbol ${}^{\textsc{\scriptsize TM}}$ 
is not permitted. Trademarked names in titles
and abstracts should be replaced by generic descriptions where
possible. If trademarked names are retained in titles and abstracts,
they will not be acknowledged as such. The first time a trademarked
name appears in the body text, it may be footnoted. The footnote will
state that it is registered, and the name of the owner.



\section{References}
References should be numbered and listed at the end of the text, and
should be cited in the text consecutively by number [shown in
square brackets].

\section{Writing Style}
Good grammar should be used and writing should be easy to
understand. Superlatives should be omitted. Words and phrases shall
not be abbreviated in titles, abstracts, nor the first time they
appear in the text.

\def\SI{$\triangle$}%
\def\deprecated{\,{\fboxsep=0pt\fbox{\vrule width0pt height0.4pc%
                                   \vrule width0.4pc height0pc}}\,}%

Metric units according to the System of International
Units (SI) should be used. Following are some frequently used SI units
and their symbols, some non-SI units that may be used with SI units
(\SI), and some non-SI units that are deprecated (\deprecated).


\def\tentry#1#2{\>#1\>#2\\}%
\begin{tabbing}
\qquad\=\kern1.75in\=ZZ\kill
\tentry{\textbf{Unit Name}}{\textbf{Unit Symbol}}
\tentry{ampere}{A}
\tentry{bit or bits}{spell out}
\tentry{bytes}{spell out}
\tentry{decibel}{dB}
\tentry{degree (plane angle) (\SI)}{$\circ$}
\tentry{farad}{F}
\tentry{gauss (\deprecated)}{Gs}
\tentry{gram}{g}
\tentry{henry}{H}
\tentry{hertz}{Hz}
\tentry{hour (\SI)}{h}
\tentry{inch (\deprecated)}{in}
\tentry{joule}{J}
\tentry{kelvin}{K}
\tentry{kilohertz}{kHz}
\tentry{kilohm}{k$\Omega$}
\tentry{liter (\SI)}{l, L}
\tentry{megahertz}{MHz}
\tentry{meter}{m}
\tentry{microfarad}{$\mu\mbox{F}$}
\tentry{micrometer}{$\mu\mbox{m}$}
\tentry{microsecond}{$\mu\mbox{s}$}
\tentry{milliampere}{mA}
\tentry{millihenry}{mH}
\tentry{millimeter}{mm}
\tentry{millivolt}{mV}
\tentry{minute (time) (\SI)}{min}
\tentry{minute (plane angle) (\SI)}{'}
\tentry{nanosecond}{ns}
\tentry{oersted (\deprecated)}{Oe}
\tentry{ohm}{$\Omega$}				
\tentry{pascal}{Pa}
\tentry{picofarad}{pF}
\tentry{second (time)}{s}
\tentry{second (plane angle) (\SI)}{"}
\tentry{siemens}{S}
\tentry{tesla}{T}
\tentry{volt}{V}
\tentry{watt}{W}
\tentry{weber}{Wb}
\end{tabbing}



\section{COPYRIGHT}
The top portion of the first page of this convention paper is
copyrighted by the Audio Engineering Society and may not be reproduced
without permission. Copyright to the content of an AES convention
paper remains with its author. However, when submitting a paper for
presentation at an AES convention, an author agrees that the AES
\textit{Journal} will have the first opportunity to consider it for
publication. If it is accepted for publication in the
\textit{Journal}, authors will be asked to execute a transfer of
copyright agreement.



\section{LAYOUT GEOMETRY}
Since you will be submitting this paper electronically, you don't have
to be concerned about using A4 paper or 8.5~by 11~inch paper. With the
current margin settings you will be able to print on either size paper
for proofing purposes before submitting your paper electronically.

\section{FIGURES}
Figures, diagrams, and charts must be dark enough to reproduce in
black only. If you are using color in a graph or chart to designate
meaning, make sure that the meaning is not lost when your paper is
printed on a B\&W printer. Do this by using redundant text to identify
color lines. Minimum line width is 1/2 point. Figures should fit into
either one (see Fig.~1) or two columns. 



Figures, tables, and illustrations should be placed consecutively in
the text, as close as possible to their reference in the text. Figures
must be normally oriented on the page and not turned sideways. All
drawings and figures must be numbered (Fig.~1 etc.) and captioned.

\begin{figure}
\vbox{\fbox{%
\dimen0=\columnwidth
\advance\dimen0 by -\fboxsep
\advance\dimen0 by -\fboxsep
\advance\dimen0 by -\columnsep
\parbox[t]{\dimen0}{%
\bigskip
\begin{center}
\end{center}
\bigskip
}}
\nobreak
\caption[1]{This is a one-column figure.}}
\end{figure}



Figure labels must be legible, at least 7 point size, using Helvetica
font for preference. Filled areas should use screens of dots or
cross-hatching rather than grey fills.

Photographs and graphic images should be saved as low-resolution
(72~dpi) files and embedded in the text following the same guidelines
as those outlined above for figures and tables.

\begin{figure*}[tb!]
\begin{center}
\fbox{\vrule width0pt height 1in\vrule width6in height 0in}
\end{center}
\caption[2]{It is best to place to place two-column figures at the bottom
or top of a page.}
\end{figure*}

\section{EQUATIONS}
Long equations are more readable when compressed into a single column
in a larger text size:
\begin{large}
\begin{eqnarray*}
X &=& a + b + c + d + e + f + g + h + i + j \\
  &&    + k  + l + m + \dots z
\end{eqnarray*}%
\end{large}%
Equations that cannot be conveniently compressed into a single column
can be extended across the entire page. It is best to put such long
equations at the top or the bottom of the page so as not to interfere
with the rest of the two-column text on the page.
\begin{equation}
x=y
\end{equation}

\section{PDF FILES}
Be certain that when creating the PostScript or PDF file from your
\LaTeX\ files that you have the PostScript Type~1 fonts installed in
your \TeX\ system. If you do not and are instead using the bitmapped
versions of the \TeX\ fonts, then your resulting PostScript/PDF file
will look very bad when printed or viewed on-line. To get the free
\TeX\ Type~1 fonts, go to 
http://www.ams.org/tex/type1-fonts.html. 
For instruction on how to install these fonts for your \TeX\ system,
contact your system administrator or \TeX\ vendor. 


\begin{thebibliography}{99}

\bibitem{DEK1}
Author, ``Title,'' presented at the AES 114th convention, 
Amsterdam, The Netherlands, 2003 March 22--25. 

%\bibitem{DEK2}
%D. E. Knuth, {\it Selected papers on analysis of algorithms}, CSLI
%Publ., Stanford, CA, 2000; CNO
%CMP 1 762 319 
%
%\bibitem{DEK3}
%D. E. Knuth, Algorithmica {\bf 22} (1998), no.~4, 561--568; MR
%2000j:68037 
%
%\bibitem{DEK4}
%R. L. Graham, D. E. Knuth and O. Patashnik, {\it Concrete mathematics}
%(Polish), Translated from the
%second English (1994) edition by P. Chrzastowski, A. Czumaj,
%L. Gasieniec and M. Raczunas, Second
%edition, Wydawnictwo Naukowe PWN, Warsaw, 1998; MR 99m:68002
%
\end{thebibliography}


\end{document}