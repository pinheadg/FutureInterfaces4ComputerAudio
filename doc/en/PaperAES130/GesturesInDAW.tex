\documentclass{aes130}
\hyphenation{Post-Script}

\usepackage{array}
\usepackage{rotating}

\usepackage{url}

\authors{Wincent Balin\aff{1} and
         J\"{o}rn Loviscach\aff{2}}
\affiliation[1]{Universit\"{a}t Oldenburg, D-26111-Oldenburg, Germany}
\affiliation[2]{Fachhochschule Bielefeld (University of Applied Sciences), D-33602-Bielefeld, Germany}
\title{Gestures to Operate DAW Software}

\lastnames{Balin, Loviscach}
\correspondence{J\"{o}rn Loviscach}{joern.loviscach@hs-bielefeld.de}

\begin{abstract}
There is a noticeable absence of gestures---be they mouse-based or
(multi-)touch-based---in mainstream digital audio workstation (DAW)
software. As an example for such a gesture consider a clockwise O
drawn with the finger to increase a value of a parameter. The increasing
availability of devices such as smartphones, tablet computers and touchscreen
displays raises the question in how far audio software can benefit from gestures.
We describe design strategies to create a consistent set of gesture commands.
The main part of this paper reports on a user survey on mappings between 22~DAW
functions and 30~single-point as well as multi-point gestures.
We discuss the findings and point out consequences for user-interface design.
\end{abstract}

\begin{document}
\maketitle

\section{Introduction}

In the last years, the concept of gesture became pervasive -- everyone in the western world either owns
at least one electronic device with gestural interface or knows someone who does. The mentioned devices
are most often portable. On non-portable devices, mainly personal computers (PC), the support of gestures
only recently began to win hearts and minds, as is the case with multi-touch gestures, or has been largely neglected,
as the situation with mouse gestures is.

Despite the circumstances described, many a power user took mouse gestures to his heart. The Opera web browser
introduced support of mouse gestures in April 2001\cite{WikipediaMouseGesture}, and this feature
had enough appeal for it's users to be reintroduced both to Mozilla Firefox (with -- currently -- more
than ten extensions\cite{FirefoxGestureExtensions}) and to Microsoft internet Explorer
(for example \cite{InternetExplorerGestures}) web browsers. Autodesk Maya, a 3D modeller software,
goes even further by using marking menus\cite{Kurtenbach:1994:ULP:259963.260376}, an input method related
to mouse gestures, by default.

But classic digital audio workstation (DAW) programs seem to shun gestures alltogether. Today,
most of the music software with gesture support runs on Apple iOS based platforms. A good example is
StepPolyArp\cite{StepPolyArp}, a MIDI arpeggiator software.
An additional event therein is created by touching an empty cell in the events grid\cite{StepPolyArpManual},
changing length of the event is done simply by dragging it, tap-and-hold and dragging afterwards is used to move the event
and to delete the event, the user places his finger upon the event and then swipes it down (akin to swiping a bead on an abacus).

Looking back to classic DAW software, there is a performance tool for Ableton Live software called Aggregat\cite{Aggregat}.
It is still under heavy development, but on the associated web page the developers provide a demonstration video.
Aggregat uses multi-touch capabilities at great lengths. The usual Ableton Live widgets are appropriately presented
on a touch-sensitive table. But aside from Aggregat, neither mouse gestures nor multi-touch gestures are used
in traditional DAW programs, and even single-touch gestures resemble nothing more than a mouse pointer.

The usual workflow when editing an audio document, as this is the main purpose of classic DAW programs, means chains
of edits. An edit means in this case the following sequence of steps:

\begin{enumerate}
\item Moving the cursor to the tool panel
\item Choosing the needed tool (through a menu or a button widget)
\item Moving the cursor back to the editing area
\item Performing an action with a tool
\item If the tool is destructive, most probably moving back to tool panel and \ldots
\item choosing a non-destructive tool
\end{enumerate}

Even without using quantification methods, such as GOMS\cite{Card:1983:PHI:578027}, it is evident
that the list above lasts much longer than a single gesture. In case the user tries to remedy the duration
of an edit through keyboard shortcuts, a gesture, especially on a touch screen, does not require to shift the user's
focus, at least occasionally, from screen to keyboard and back. Also, keyboard shortcuts require some time
and practice to become memorized. This is less true for gestures\cite{Appert:2009:USC:1518701.1519052}.

To conclude, which gestures are most appropriate for which functions of DAW software, we conducted
a survey among trained DAW users. First, we selected common DAW functions, performed repeatedly and often.
Second, we compiled a collection of gestures and selected 30 of them to appear in the survey. Then
we presented a bilingual survey to DAW users, who chose, which gesture does correspond to which DAW function
the most.

This paper is structured around the survey. The next section presents related approaches.
Section~\ref{sec:EditingTasks} discusses which audio editing tasks will be taken into the survey.
The section~\ref{sec:Gestures} describes available gestures and the process of their selection.
The main sections~\ref{sec:Survey1} and \ref{sec:Survey2} discuss the first and the second versions of the survey.
Eventually, the section~\ref{sec:Conclusions} concludes this paper.

\section{Related work}

Using gestures to control DAW software touches many different aspects.

It should be said specifically about the notion of \emph{intuitive} gestures that there is a difference
between the gestures intuitive for a beginner and those intuitive for an expert. Whereas the concept of
intuition implies \emph{non-concious} use of knowledge \cite{Naumann:2007:IUU:1784197.1784212}, an expert
has additional layers of knowledge, acquired by experience. Our study 

\cite{Appert:2009:USC:1518701.1519052}

%The differences between the mouse and the multi-touch interface, from which the differences between
%the mouse gestures and multi-touch gestures will be expanded, were introduced in \cite{Buxton:1985:ITT:325334.325239}.
%Such differences consist of 
%
%A graphic similarity between mouse gestures and marking menus, as well several statements related
%to gestures, were demonstrated in \cite{Kurtenbach:1993:LEP:164632.164977}.
%The differences between mouse and multi-touch screen, contrary to differences between mouse input
%and touch tablet, as it was in \cite{Kurtenbach:1993:LEP:164632.164977}, are discussed in
%\cite{Forlines:2007:DVM:1240624.1240726}.
%
%The taxonomy in \cite{ecs11149} helps to classify the groups of gestures observed in this paper.
%A small taxonomy for multi-touch gestures is also provided in \cite{Wigdor:2010:ANU:1842993.1842997},
%but it is directed towards surface input devices. Another taxonomy is presented 
%n \cite{Wobbrock:2009:UGS:1518701.1518866}; also, this work has many useful observations,
%some of which were seconded in the results of the survey.
%
%When choosing gesture candidates for the survey, the paper \cite{Naumann:2007:IUU:1784197.1784212}
%explains the concept of an ``intuitive'' gesture in a concise manner. The document \cite{Lee:2009:EEP:1520340.1520667}
%helps to counter the overuse of gestures with inertial behaviour, as well as distinguishing between
%the needs of mouse gestures and touch gestures. Another work that cautions against improvident use
%of gestures is \cite{Norman:2010:GIS:1836216.1836228}.
%
%The methodology of designing a gestural language using brainstorming is explained in
%\cite{Akers:2007:ODM:1240866.1240868}. In a similar venue lies \cite{SalatBottomUpApproach}, which explains
%the bottom-up approach to software design.
%
%A design both of a gesture language and of a software package, which connects the gestures
%as a replacement or an addition to applications written in Java programming language and using
%its Swing toolkit, is presented in \cite{Appert:2009:USC:1518701.1519052}. The document also
%provides a comparison between mouse gestures and keyboard shortcuts.
%
%The paper \cite{Kammer:2010:TFM:1936652.1936662} describes formal description for multi-touch gestures,
%which is also capable of producing formal descriptions of single-touch as well as mouse gestures.

\section{Selection of editing tasks} \label{sec:EditingTasks}



\section{Selection of gestures} \label{sec:Gestures}

% Include graphics
\newcommand{\quarterpic}[1][]{\includegraphics[scale=0.25]{../../de/Befragung/img/#1}}
\newcommand{\sixthpic}[1][]{\includegraphics[scale=0.17]{../../de/Befragung/img/#1}}

Definition of gesture

Taxonomy: one-time/continuous/repeated

Direction (vertical/horizontal) of a (zoom-)gesture matters

Static/dynamic

Using finger nails

StrokeIt


\section{First survey} \label{sec:Survey1}


\subsection{Results} \label{sec:Survey1Results}

\begin{table} \label{tab:Survey1RateOfChoices}
\begin{center}
\begin{tabular}{|c|r||c|r||c|r|} \hline

\sixthpic[right]            & 14 & \sixthpic[up]           & 12 & \sixthpic[rotate-clockwise]              & 24 \\ \hline
\sixthpic[right-up]         &  2 & \sixthpic[up-left]      & 10 & \sixthpic[rotate-counterclockwise]       & 16 \\ \hline
\sixthpic[right-down]       & 17 & \sixthpic[up-right]     &  9 & \sixthpic[digit-1]                       &  9 \\ \hline
\sixthpic[right-left]       &  5 & \sixthpic[up-down]      &  8 & \sixthpic[letter-a]                      & 33 \\ \hline
\sixthpic[right-left-right] &  7 & \sixthpic[up-down-up]   &  5 & \sixthpic[multi-pinch-horizontal]        & 15 \\ \hline
\sixthpic[left]             & 10 & \sixthpic[down]         & 17 & \sixthpic[multi-pinch-vertical]          &  7 \\ \hline
\sixthpic[left-up]          &  6 & \sixthpic[down-left]    &  5 & \sixthpic[multi-spread-horizontal]       & 21 \\ \hline
\sixthpic[left-down]        &  2 & \sixthpic[down-right]   &  9 & \sixthpic[multi-spread-vertical]         & 15 \\ \hline
\sixthpic[left-right]       & 11 & \sixthpic[down-up]      &  9 & \sixthpic[multi-rotate-clockwise]        &  7 \\ \hline
\sixthpic[left-right-left]  & 13 & \sixthpic[down-up-down] &  5 & \sixthpic[multi-rotate-counterclockwise] &  6 \\ \hline

\end{tabular}
\end{center}
\caption{Choice frequency for different gestures}
\end{table}


% Make vertical cells
\newcolumntype{v}[1]{%
  >{\begin{turn}{90}\begin{minipage}{15em}\raggedright\hspace{0pt}}l%
  <{\end{minipage}\end{turn}}|%
}
\newcolumntype{w}[1]{%
  >{\begin{turn}{90}\begin{minipage}{5em}\raggedright\hspace{0pt}}l%
  <{\end{minipage}\end{turn}}|%
}

\begin{table*} \label{tab:Survey1Results}
\footnotesize
\begin{tabular}{|c|c|c|c|c|c|c|c|c|c|c|c|c|c|c|c|c|c|c|c|c|c|c|} \hline

 & \multicolumn{1}{v|}{Go one bar forward}
 & \multicolumn{1}{v|}{Go one bar back}
 & \multicolumn{1}{v|}{Jump to beginning}
 & \multicolumn{1}{v|}{Jump to end}
 & \multicolumn{1}{v|}{Toggle metronome}
 & \multicolumn{1}{v|}{Select all}
 & \multicolumn{1}{v|}{Cut selected region}
 & \multicolumn{1}{v|}{Copy selected region}
 & \multicolumn{1}{v|}{Paste copied or cut region}
 & \multicolumn{1}{v|}{Duplicate selected region}
 & \multicolumn{1}{v|}{Delete selected region}
 & \multicolumn{1}{v|}{Split selected region}
 & \multicolumn{1}{v|}{Glue selected regions} 
 & \multicolumn{1}{v|}{Undo last action}
 & \multicolumn{1}{v|}{Redo action undone last}
 & \multicolumn{1}{v|}{Increase value of a control}
 & \multicolumn{1}{v|}{Decrease value of a control}
 & \multicolumn{1}{v|}{Fit selected track to window}
 & \multicolumn{1}{v|}{Fit selected region to window}
 & \multicolumn{1}{v|}{Fit all tracks to window: vert.}
 & \multicolumn{1}{v|}{Fit all tracks to window: horiz.}
 & \multicolumn{1}{v|}{Recall zoom preset} \\ \hline \hline

\multicolumn{1}{|w|}{Number of answers}
 & 17 & 18 & 18 & 17 & 11 & 17 & 14 & 14 & 16 & 16 & 14 & 16 & 14 & 18 & 18 & 16 & 15 & 11 & 14 & 14 & 14 & 13 \\ \hline

\sixthpic[right]
 & 59 &  &  & 12 &  &  &  &  &  &  &  &  &  &  & 6 &  &  & 9 &  &  &  &  \\ \hline

\sixthpic[right-up]
 &  &  &  &  &  &  & 7 &  &  &  &  &  &  &  &  &  &  &  & 7 &  &  &  \\ \hline

\sixthpic[right-down]
 & 6 &  &  & 35 &  &  &  & 14 & 25 & 12 & 7 &  &  &  &  &  & 7 &  &  &  &  &  \\ \hline

\sixthpic[right-left]
 & 6 &  & 17 &  &  &  &  &  &  &  &  & 6 &  &  &  &  &  &  &  &  &  &  \\ \hline

\sixthpic[right-left-right]
 &  &  &  & 6 &  & 6 &  &  &  &  & 7 &  &  &  & 6 &  &  &  & 7 &  & 14 &  \\ \hline

\sixthpic[left]
 &  & 11 & 28 &  &  &  &  &  &  &  & 7 &  &  & 11 &  &  &  &  &  &  &  &  \\ \hline

\sixthpic[left-up]
 &  &  & 6 &  &  &  & 7 & 7 &  &  &  &  &  &  & 11 & 6 &  &  &  &  &  &  \\ \hline

\sixthpic[left-down]
 &  & 6 &  &  &  &  &  &  & 6 &  &  &  &  &  &  &  &  &  &  &  &  &  \\ \hline

\sixthpic[left-right]
 &  & 6 &  & 24 &  &  & 7 &  &  &  &  &  & 7 &  & 11 &  &  &  &  &  & 7 & 8 \\ \hline

\sixthpic[left-right-left]
 &  &  & 6 &  &  &  &  &  &  &  & 21 &  &  & 22 & 6 &  &  & 18 & 7 &  & 7 &  \\ \hline

\sixthpic[up]
 &  &  & 6 &  &  &  & 7 &  &  &  &  &  & 7 &  &  & 56 &  &  &  &  &  &  \\ \hline

\sixthpic[up-left]
 &  & 17 & 22 &  & 9 &  &  &  &  &  &  &  &  & 11 &  &  &  &  &  &  &  &  \\ \hline

\sixthpic[up-right]
 & 12 &  &  & 12 &  &  &  & 14 &  & 6 &  &  &  & 6 & 6 &  &  &  &  &  &  &  \\ \hline

\sixthpic[up-down]
 &  &  &  &  &  &  & 7 & 14 &  & 19 &  & 6 &  &  & 6 &  &  &  &  &  &  &  \\ \hline

\sixthpic[up-down-up]
 &  &  &  &  &  &  &  & 7 &  & 6 &  &  &  &  &  &  &  &  & 7 & 7 &  & 8 \\ \hline

\sixthpic[down]
 &  &  &  & 6 &  &  &  & 7 & 19 &  & 14 & 12 &  &  &  &  & 53 &  &  &  &  &  \\ \hline

\sixthpic[down-left]
 &  & 11 & 11 &  &  &  &  &  &  &  &  &  &  &  &  &  &  &  &  &  &  & 8 \\ \hline

\sixthpic[down-right]
 & 6 &  &  &  & 9 &  &  &  & 19 & 12 & 7 &  &  &  & 6 &  &  &  &  &  &  &  \\ \hline

\sixthpic[down-up]
 &  &  &  &  & 9 &  &  &  & 6 & 6 &  & 19 &  &  &  &  &  & 9 & 7 & 7 &  &  \\ \hline

\sixthpic[down-up-down]
 &  &  &  &  &  &  & 7 &  & 6 &  &  &  &  &  &  &  &  &  &  & 21 &  &  \\ \hline

\sixthpic[rotate-clockwise]
 &  & 11 &  & 6 &  & 35 &  & 14 &  & 12 & 7 &  &  & 11 & 17 & 25 &  &  &  &  &  & 8 \\ \hline

\sixthpic[rotate-counterclockwise]
 & 12 &  &  &  &  & 18 &  &  &  & 6 &  &  & 7 & 17 & 6 &  & 29 &  &  &  & 7 &  \\ \hline

\sixthpic[digit-1]
 &  &  &  &  & 18 & 6 & 7 &  & 6 &  & 7 &  &  & 6 &  &  &  &  &  &  &  & 15 \\ \hline

\sixthpic[letter-a]
 &  &  &  &  & 55 & 29 & 29 & 14 & 6 & 12 & 7 & 6 &  & 11 & 11 &  &  & 9 & 7 & 7 & 7 & 15 \\ \hline

\sixthpic[multi-pinch-horizontal]
 &  &  &  &  &  & 6 & 7 & 7 &  &  & 7 &  & 57 &  &  &  &  &  &  &  & 14 & 8 \\ \hline

\sixthpic[multi-pinch-vertical]
 &  &  &  &  &  &  &  &  &  &  &  &  & 14 &  &  &  &  & 18 & 7 & 14 &  &  \\ \hline

\sixthpic[multi-spread-horizontal]
 &  &  &  &  &  &  & 7 &  &  &  & 7 & 37 &  &  &  &  &  & 27 & 29 &  & 43 &  \\ \hline

\sixthpic[multi-spread-vertical]
 &  &  &  &  &  &  & 7 &  &  &  &  & 12 &  &  &  &  &  &  & 21 & 43 &  & 23 \\ \hline

\sixthpic[multi-rotate-clockwise]
 &  &  &  &  &  &  &  &  &  & 6 &  &  & 7 &  & 11 & 12 &  & 9 &  &  &  &  \\ \hline

\sixthpic[multi-rotate-counterclockwise]
 &  &  & 6 &  &  &  &  &  & 6 &  &  &  &  & 6 &  &  & 13 &  &  &  &  & 8 \\ \hline

\end{tabular}
\caption{Results of the first survey (in \emph{per cent}, rounded)}
\end{table*}


\subsection{Observations} \label{sec:SurveyObservations}


\begin{table} \label{tab:Survey1CumulativePopularity}
\begin{center}
\begin{tabular}{|c|r||c|r||c|r|} \hline

\sixthpic[right]            &  86 & \sixthpic[up]           &  76 & \sixthpic[rotate-clockwise]              & 146 \\ \hline
\sixthpic[right-up]         &  14 & \sixthpic[up-left]      &  59 & \sixthpic[rotate-counterclockwise]       & 102 \\ \hline
\sixthpic[right-down]       & 106 & \sixthpic[up-right]     &  56 & \sixthpic[digit-1]                       &  65 \\ \hline
\sixthpic[right-left]       &  29 & \sixthpic[up-down]      &  52 & \sixthpic[letter-a]                      & 225 \\ \hline
\sixthpic[right-left-right] &  46 & \sixthpic[up-down-up]   &  35 & \sixthpic[multi-pinch-horizontal]        & 106 \\ \hline
\sixthpic[left]             &  57 & \sixthpic[down]         & 111 & \sixthpic[multi-pinch-vertical]          &  53 \\ \hline
\sixthpic[left-up]          &  37 & \sixthpic[down-left]    &  30 & \sixthpic[multi-spread-horizontal]       & 150 \\ \hline
\sixthpic[left-down]        &  12 & \sixthpic[down-right]   &  59 & \sixthpic[multi-spread-vertical]         & 106 \\ \hline
\sixthpic[left-right]       &  70 & \sixthpic[down-up]      &  63 & \sixthpic[multi-rotate-clockwise]        &  45 \\ \hline
\sixthpic[left-right-left]  &  87 & \sixthpic[down-up-down] &  34 & \sixthpic[multi-rotate-counterclockwise] &  39 \\ \hline

\end{tabular}
\end{center}
\caption{Cumulative popularity of gestures in the first survey}
\end{table}


\begin{table} \label{tab:Survey1AverageGroups}
\begin{center}
\begin{tabular}{|l|r|} \hline

Linear gestures starting rightwards &  56 \\ \hline
Linear gestures starting leftwards  &  53 \\ \hline
Linear gestures starting upwards    &  56 \\ \hline
Linear gestures starting downwards  &  59 \\ \hline \hline

Unidirectional gestures             &  82 \\ \hline
$90^\circ$ gestures                 &  47 \\ \hline
Bidirectional gestures              &  53 \\ \hline
Tridirectional gestures             &  50 \\ \hline \hline

Rotate gestures                     & 124 \\ \hline \hline

Letter gestures                     & 225 \\ \hline
Digit gestures                      &  65 \\ \hline \hline

Multi-touch pinch gestures          &  79 \\ \hline
Multi-touch spread gestures         & 128 \\ \hline
Multi-touch rotate gestures         &  42 \\ \hline

\end{tabular}
\end{center}
\caption{Average popularity of gesture groups}
\end{table}


\section{Second survey} \label{sec:Survey2}

\subsection{Changes} \label{sec:Survey2Changes}

\subsection{Results} \label{sec:Survey2Results}

\subsection{Observations} \label{sec:Survey2Observations}



\section{Conclusions} \label{sec:Conclusions}


%It is interesting to see how persistent the usage of button widgets in DAW software became:
%Even in the recently unveiled Garageband for iPad DAW the playback is controlled by three
%buttons\cite{GaragebandIpadDemo} (Jump to the beginning/Stop, Play/Pause and Record),
%as in spite of running on a multi-touch enabled platform. 

Accelerometer used for velocity-sensitivity in Garageband for iPad\cite{GaragebandIpadDemo}.

Conducting using Kinect\cite{KinectConductingGameIdea}

\section{Acknowledgements} \label{sec:Acknowledgements}

\bibliographystyle{abbrv}
\bibliography{GesturesInDAW}

\end{document}
