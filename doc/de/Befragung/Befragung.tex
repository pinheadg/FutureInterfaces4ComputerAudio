\documentclass[11pt,a4paper,notitlepage]{article}

% Input encoding
\usepackage[utf8]{inputenc}
\usepackage[T1]{fontenc}  		% nur gemeinsam mit lmodern!
\usepackage{lmodern}
\usepackage{ngerman}

\usepackage{amsmath}
\usepackage{amsfonts}
\usepackage{amssymb}

\usepackage{booktabs}

\usepackage[pdftex]{graphicx}

\usepackage[pdfauthor={Wincent Balin},
            pdftitle={Befragung zur Bedienung von Pro Tools mit Mausgesten}]{hyperref}

\author{Wincent Balin}
\title{Befragung zur Bedienung von \emph{Pro Tools} mit~Mausgesten}

\begin{document}
\maketitle

Die Befragung wird durchgeführt, um für die Diplomarbeit
\textsf{Bedienschnittstellen für Audioverarbeitung: Randbedingungen und ungenutzte Möglichkeiten},
genauer für das Unterthema \textsf{Bedienung von DAWs mit Gesten} Vorschläge zu sammeln.

\begin{table}[ht] \label{tab:Gestures}
\centering
\begin{tabular}{lcl} \toprule
Nummer & Darstellung & Erklärung \\ \midrule \midrule
 1 & \includegraphics[scale=0.25]{img/up} & \\ \midrule
\end{tabular}
\caption{Verfügbare Gesten}
\end{table}

Tragen Sie bitte in die nachfolgende Tabelle die Nummern der Ihrer Ansicht nach geeignetsten Gesten
in die leere Spalte ein. Einträge von mehreren Gestennummern sind erlaubt. Beim Eintragen von mehreren
Gestern gilt die zuerst eingetragene als die am meisten geeignete, die nachfolgende als die weniger geeignete
und so weiter.

\begin{table}[ht] \label{tab:Functions}
\centering
\begin{tabular}{llll} \toprule
Deutsche     & Englische    & Tasten-  & Gesten-        \\
Bezeichnung  & Bezeichnung  & kombina- & vorschlag      \\
der Funktion & der Funktion & tion     & (ggf. mehrere) \\ \midrule \midrule
\end{tabular}
\caption{Tabelle der \emph{Pro Tools}-Funktionen}
\end{table}

\end{document}
