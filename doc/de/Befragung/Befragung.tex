\documentclass[11pt,a4paper,notitlepage]{article}

% Input encoding
\usepackage[utf8]{inputenc}
\usepackage[T1]{fontenc}  		% nur gemeinsam mit lmodern!
\usepackage{lmodern}
\usepackage{ngerman}

\usepackage{amsmath}
\usepackage{amsfonts}
\usepackage{amssymb}

\usepackage{booktabs}
\usepackage{supertabular}

\usepackage[pdftex]{graphicx}

\usepackage[pdfauthor={Wincent Balin},
            pdftitle={Befragung zur Bedienung von Pro Tools mit Mausgesten}]{hyperref}

\author{Wincent Balin}
\title{Befragung\\zur Bedienung von \emph{Pro Tools}\\mit Mausgesten}

\begin{document}
\maketitle


% Scale picture to 0.25 automagically
\newcommand{\quarterpic}[1][]{\includegraphics[scale=0.25]{img/#1}}

Stellen Sie sich vor: Sie sitzen vor Ihrem Arbeitsrechner und haben Ihr bevorzugtes Audiobearbeitungsprogramm
auf dem Bildschirm, aber die Tastatur fehlt. Sie können das Programm mit der Maus oder mit einem Touchscreen bedienen.
Nach einer Weile vermissen Sie die Tastenkürzel und möchten diese durch Maus- oder Touchscreengesten ersetzen.
Die Mausgesten zeichnen Sie mit der gedrückten rechten Maustaste.

Wählen Sie in der nachfolgenden Umfrage die Ihrer Meinung nach zu den genannten Funktionen passenden Gesten.
Der rote Punkt steht für den Anfang einer Geste. Mehrere Punkte bedeuten eine Multitouchgeste.
Die Geste \quarterpic[digit-1] ist stellvertretend für die mit dem Eingabegerät gezeichneten Ziffern. Genauso steht
die Geste \quarterpic[letter-a] für die mit dem Eingabegerät geschriebenen Buchstaben.

% Enumerate table rows
\newcounter{rownum}
\newcommand{\rownumber}{\addtocounter{rownum}{1}\arabic{rownum}}
\setcounter{rownum}{0}

\tablelasttail{\bottomrule}

\begin{center}
\tablefirsthead
{
  \toprule
  Nummer & Darstellung & Erklärung \\ \cmidrule(r){1-1} \cmidrule(lr){2-2} \cmidrule(l){3-3}
}
\tablehead
{
  \toprule
  \multicolumn{3}{@{}l}{\small Fortsetzung} \\ \midrule
  Nummer & Darstellung & Erklärung \\ \cmidrule(r){1-1} \cmidrule(lr){2-2} \cmidrule(l){3-3}
}
\tabletail
{
  \midrule
  \multicolumn{3}{@{}r}{\small Fortsetzung folgt} \\ \bottomrule
}
\bottomcaption{Verfügbare Gesten}
\begin{supertabular}{rcl}
  \rownumber & \quarterpic[right] & Bewegung nach rechts \\
  \rownumber & \quarterpic[right-up] & Bewegung nach rechts, dann nach oben \\
  \rownumber & \quarterpic[right-down] & Bewegung nach rechts, dann nach unten \\
  \rownumber & \quarterpic[right-left] & Bewegung nach rechts, dann nach links \\
  \rownumber & \quarterpic[right-left-right] & Bewegung nach rechts, dann nach links, rechts \\
  \rownumber & \quarterpic[left] & Bewegung nach links \\
  \rownumber & \quarterpic[left-up] & Bewegung nach links, dann nach oben \\
  \rownumber & \quarterpic[left-down] & Bewegung nach links, dann nach unten \\
  \rownumber & \quarterpic[left-right] & Bewegung nach links, dann nach rechts \\
  \rownumber & \quarterpic[left-right-left] & Bewegung nach links, dann nach rechts, links \\
  \rownumber & \quarterpic[up] & Bewegung nach oben \\
  \rownumber & \quarterpic[up-left] & Bewegung nach oben, dann nach links \\
  \rownumber & \quarterpic[up-right] & Bewegung nach oben, dann nach rechts \\
  \rownumber & \quarterpic[up-down] & Bewegung nach oben, dann nach unten \\
  \rownumber & \quarterpic[up-down-up] & Bewegung nach oben, dann nach unten, oben \\
  \rownumber & \quarterpic[down] & Bewegung nach unten \\
  \rownumber & \quarterpic[down-left] & Bewegung nach unten, dann nach links \\
  \rownumber & \quarterpic[down-right] & Bewegung nach unten, dann nach rechts \\
  \rownumber & \quarterpic[down-up] & Bewegung nach unten, dann nach oben \\
  \rownumber & \quarterpic[down-up-down] & Bewegung nach unten, dann nach oben, unten \\
  \rownumber & \quarterpic[diag-up-left] & Bewegung diagonal nach oben links \\
  \rownumber & \quarterpic[diag-up-left-down-right] & Bewegung diagonal nach oben links und zurück \\
  \rownumber & \quarterpic[diag-up-right] & Bewegung diagonal nach oben rechts \\
  \rownumber & \quarterpic[diag-up-right-down-left] & Bewegung diagonal nach oben rechts und zurück \\
  \rownumber & \quarterpic[diag-down-left] & Bewegung diagonal nach unten links \\
  \rownumber & \quarterpic[diag-down-left-up-right] & Bewegung diagonal nach unten links und zurück \\
  \rownumber & \quarterpic[diag-down-right] & Bewegung diagonal nach unten rechts \\
  \rownumber & \quarterpic[diag-down-right-up-left] & Bewegung diagonal nach unten rechts und zurück \\
  \rownumber & \quarterpic[rotate-clockwise] & Kreisbewegung im Uhrzeigersinn \\
  \rownumber & \quarterpic[rotate-counterclockwise] & Kreisbewegung gegen den Uhrzeigersinn \\
  \rownumber & \quarterpic[digit-1] & Eine Ziffer wird gezeichnet \\
  \rownumber & \quarterpic[letter-a] & Ein Buchstabe wird gezeichnet \\
  \rownumber & \quarterpic[multi-pinch-horizontal] & Zwei Finger horizontal zusammen ziehen \\
  \rownumber & \quarterpic[multi-pinch-vertical] & Zwei Finger vertikal zusammen ziehen \\
  \rownumber & \quarterpic[multi-spread-horizontal] & Zwei Finger horizontal auseinander ziehen \\
  \rownumber & \quarterpic[multi-spread-vertical] & Zwei Finger vertikal auseinander ziehen \\
  \rownumber & \quarterpic[multi-rotate-clockwise] & Drehung mit zwei Fingern im Uhrzeigersinn \\
  \rownumber & \quarterpic[multi-rotate-counterclockwise] & Drehung mit zwei Fingern gegen den Uhrzeigersinn \\
  \rownumber & \quarterpic[empty] &   \\
  \rownumber & \quarterpic[empty] &   \\
  \rownumber & \quarterpic[empty] &   \\
  \rownumber & \quarterpic[empty] &   \\
  \rownumber & \quarterpic[empty] &   \\
  \rownumber & \quarterpic[empty] &   \\
  \rownumber & \quarterpic[empty] &   \\
  \rownumber & \quarterpic[empty] &   \\
  \rownumber & \quarterpic[empty] &   \\
  \rownumber & \quarterpic[empty] &   \\
  \rownumber & \quarterpic[empty] &   \\
\end{supertabular}
\label{tab:Gestures}
\end{center}



\begin{itemize}
\item Einen Takt vorwärts gehen
\item Einen Takt zurück gehen
\item Zum Anfang springen
\item Zum Ende springen
\item Einzelnes Track ins Fenster einpassen
\item Markierten Abschnitt ins Fenster einpassen
\item Alle Spuren vertikal ins Fenster einpassen
\item Alle Spuren horizontal ins Fenster einpassen
\item Zoom-Voreinstellung aufrufen
\item Markierten Abschnitt ausschneiden
\item Markierten Abschnitt kopieren
\item Vorher kopierten oder ausgeschnittenen Abschnitt einfügen
\item Markierten Abschnitt duplizieren
\item Markierten Abschnitt entfernen
\item Markierten Abschnitt trennen
\item Markierte Abschnitte zusammenkleben
\item Metronom ein-/ausschalten
\item Letzte Aktion widerrufen
\item Zuletzt widerrufene Aktion wiederholen
\item Alles auswählen
\item Reglerwert erhöhen
\item Reglerwert herabsetzen
\end{itemize}

\begin{center}
\centering
\tablefirsthead
{
  \toprule
  Deutsche     & Englische    & Tasten- & Gesten-        \\
  Bezeichnung  & Bezeichnung  & kombi-  & vorschlag      \\
  der Funktion & der Funktion & nation  & (ggf. mehrere) \\ \cmidrule(r){1-1} \cmidrule(lr){2-2} \cmidrule(lr){3-3} \cmidrule(l){4-4}
}
\tablehead
{
  \toprule
  \multicolumn{4}{@{}l}{\small Fortsetzung} \\ \midrule
  Deutsche     & Englische    & Tasten- & Gesten-        \\
  Bezeichnung  & Bezeichnung  & kombi-  & vorschlag      \\
  der Funktion & der Funktion & nation  & (ggf. mehrere) \\ \cmidrule(r){1-1} \cmidrule(lr){2-2} \cmidrule(lr){3-3} \cmidrule(l){4-4}
}
\tabletail
{
  \midrule
  \multicolumn{4}{@{}r}{\small Fortsetzung folgt} \\ \bottomrule
}
\bottomcaption{\emph{Pro Tools}-Funktionen}
\begin{supertabular}{llll}
 & & & \\
\end{supertabular}
\label{tab:Functions}
\end{center}


\end{document}
